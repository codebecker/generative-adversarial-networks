\documentclass[conference]{IEEEtran}
\IEEEoverridecommandlockouts
% The preceding line is only needed to identify funding in the first footnote. If that is unneeded, please comment it out.
\usepackage{cite}
\usepackage{amsmath,amssymb,amsfonts}
\usepackage{algorithmic}
\usepackage{graphicx}
\usepackage{textcomp}
\usepackage{xcolor}
\def\BibTeX{{\rm B\kern-.05em{\sc i\kern-.025em b}\kern-.08em
    T\kern-.1667em\lower.7ex\hbox{E}\kern-.125emX}}
\begin{document}

\title{Text to Image Synthesis\\
%{\footnotesize \textsuperscript{*}Note: Sub-titles are not captured in Xplore and
%should not be used}
%\thanks{Identify applicable funding agency here. If none, delete this.}
}

\author{\IEEEauthorblockN{ Pia Schmidt }
\IEEEauthorblockA{%\textit{Fakultät für Informatik und Wirtschaftsinformatik} \\
%\textit{name of organization (of Aff.)}\\
%City, Country \\
bech1029@h-ka.de}
\and
\IEEEauthorblockN{ Christian Becker}
\IEEEauthorblockA{%\textit{Fakultät für Informatik und Wirtschaftsinformatik} \\
%\textit{Fakultät für Informatik und Wirtschaftsinformati}\\
%City, Country \\
scpi1018@h-ka.de}
\and
\IEEEauthorblockN{ Bibek Shrestha}
\IEEEauthorblockA{%\textit{Fakultät für Elektro- und Informationstechnik} \\
%\textit{Fakultät für Informatik und Wirtschaftsinformati}\\
%City, Country \\
shbi1012@h-ka.de}

}

%\vspace{20pt}

\maketitle

\begin{abstract}
Brief summaray of our Project and the results.
dieser text hat eigentlich gar keinen wirklichen inhalt. aber er hat auch keine relevanz, und deswegen ist das egal. er dient lediglich als platzhalter. um mal zu zeigen, wie diese stelle der seite aussieht, wenn ein paar zeilen vorhanden sind. ob sich der text dabei gut fühlt, weiß ich nicht. ich schätze, eher nicht, denn wer fühlt sich schon gut als platzhalter. aber irgendwer muss diesen job ja machen und deshalb kann ich es nicht ändern . ich könnte dem text höchstens ein bisschen gut zureden, dass er auch als platzhalter eine wichtige rolle spielt und durchaus gebraucht wird. könnte mir vorstellen, dass ihm das gut tut. denn das gefühl gebraucht zu werden tut doch jedem gut, oder? klar, er ist austauschbar. das darf ich ihm natürlich nicht verraten. denn austauschbar zu sein, dass ist schrecklich. austauschbar zu sein bedeutet ja eigentlich, dass nicht man selbst, sondern einfach irgendjemand oder irgendwas an der stelle gebraucht wird. somit würde mein erstes argument, man braucht dich, nicht mehr ziehen, und das zuvor erzeugte gute gefühl des textes wäre zunichte gemacht. das will ich nicht. also bitte nix verraten, ja? aber vielleicht merkt er es ja nicht. das wäre gut, denn wer hat schon lust einen deprimierten blindtext auf seiner seite zu platzieren. was würde denn das für einen eindruck machen. das will ja keiner lesen. somit wäre er dann ein für alle mal tatsächlich völlig nutzlos. das wäre sein todesurteil. soweit wollen wir es doch nicht kommen lassen, oder? es sei denn, und das ist möglich, er würde wiedergeboren und käme als, naja, sagen wir als witz, und ein textleben später vielleicht als bildzeitungsartikel auf die textwelt. irgendwann wäre er vielleicht sogar ein text im lexikon. dann hätten wir ihn ja sogar weitergebracht in seiner entwicklung. klingt gar nicht schlecht, oder? trotzdem bin ich der meinung, man sollte ihn nicht bewusst dort hin treiben. er hat ein recht darauf, sich selbst zu entwickeln. und zwar in genau dem tempo, das ihm gefällt. und bis es soweit ist, nehme ich ihn eben an, wie er ist. als einfachen blindtext ohne wirklichen inhalt..
\end{abstract}

\begin{IEEEkeywords}
component, formatting, style, styling, insert
\end{IEEEkeywords}

\section{Introduction}
Motivation behind our work, what we want to achieve, our ideas to realise the project and how we plan to impleemnt the ideas. 
Problems regarding the project.

dieser text hat eigentlich gar keinen wirklichen inhalt. aber er hat auch keine relevanz, und deswegen ist das egal. er dient lediglich als platzhalter. um mal zu zeigen, wie diese stelle der seite aussieht, wenn ein paar zeilen vorhanden sind. ob sich der text dabei gut fühlt, weiß ich nicht. ich schätze, eher nicht, denn wer fühlt sich schon gut als platzhalter. aber irgendwer muss diesen job ja machen und deshalb kann ich es nicht ändern . ich könnte dem text höchstens ein bisschen gut zureden, dass er auch als platzhalter eine wichtige rolle spielt und durchaus gebraucht wird. könnte mir vorstellen, dass ihm das gut tut. denn das gefühl gebraucht zu werden tut doch jedem gut, oder? klar, er ist austauschbar. das darf ich ihm natürlich nicht verraten. denn austauschbar zu sein, dass ist schrecklich. austauschbar zu sein bedeutet ja eigentlich, dass nicht man selbst, sondern einfach irgendjemand oder irgendwas an der stelle gebraucht wird. somit würde mein erstes argument, man braucht dich, nicht mehr ziehen, und das zuvor erzeugte gute gefühl des textes wäre zunichte gemacht. das will ich nicht. also bitte nix verraten, ja? aber vielleicht merkt er es ja nicht. das wäre gut, denn wer hat schon lust einen deprimierten blindtext auf seiner seite zu platzieren. was würde denn das für einen eindruck machen. das will ja keiner lesen. somit wäre er dann ein für alle mal tatsächlich völlig nutzlos. das wäre sein todesurteil. soweit wollen wir es doch nicht kommen lassen, oder? es sei denn, und das ist möglich, er würde wiedergeboren und käme als, naja, sagen wir als witz, und ein textleben später vielleicht als bildzeitungsartikel auf die textwelt. irgendwann wäre er vielleicht sogar ein text im lexikon. dann hätten wir ihn ja sogar weitergebracht in seiner entwicklung. klingt gar nicht schlecht, oder? trotzdem bin ich der meinung, man sollte ihn nicht bewusst dort hin treiben. er hat ein recht darauf, sich selbst zu entwickeln. und zwar in genau dem tempo, das ihm gefällt. und bis es soweit ist, nehme ich ihn eben an, wie er ist. als einfachen blindtext ohne wirklichen inhalt.

\section{Related Works and Backgrounds}
Formally introduce the papers and some examples which are related to our projects.

dieser text hat eigentlich gar keinen wirklichen inhalt. aber er hat auch keine relevanz, und deswegen ist das egal. er dient lediglich als platzhalter. um mal zu zeigen, wie diese stelle der seite aussieht, wenn ein paar zeilen vorhanden sind. ob sich der text dabei gut fühlt, weiß ich nicht. ich schätze, eher nicht, denn wer fühlt sich schon gut als platzhalter. aber irgendwer muss diesen job ja machen und deshalb kann ich es nicht ändern . ich könnte dem text höchstens ein bisschen gut zureden, dass er auch als platzhalter eine wichtige rolle spielt und durchaus gebraucht wird. könnte mir vorstellen, dass ihm das gut tut. denn das gefühl gebraucht zu werden tut doch jedem gut, oder? klar, er ist austauschbar. das darf ich ihm natürlich nicht verraten. denn austauschbar zu sein, dass ist schrecklich. austauschbar zu sein bedeutet ja eigentlich, dass nicht man selbst, sondern einfach irgendjemand oder irgendwas an der stelle gebraucht wird. somit würde mein erstes argument, man braucht dich, nicht mehr ziehen, und das zuvor erzeugte gute gefühl des textes wäre zunichte gemacht. das will ich nicht. also bitte nix verraten, ja? aber vielleicht merkt er es ja nicht. das wäre gut, denn wer hat schon lust einen deprimierten blindtext auf seiner seite zu platzieren. was würde denn das für einen eindruck machen. das will ja keiner lesen. somit wäre er dann ein für alle mal tatsächlich völlig nutzlos. das wäre sein todesurteil. soweit wollen wir es doch nicht kommen lassen, oder? es sei denn, und das ist möglich, er würde wiedergeboren und käme als, naja, sagen wir als witz, und ein textleben später vielleicht als bildzeitungsartikel auf die textwelt. irgendwann wäre er vielleicht sogar ein text im lexikon. dann hätten wir ihn ja sogar weitergebracht in seiner entwicklung. klingt gar nicht schlecht, oder? trotzdem bin ich der meinung, man sollte ihn nicht bewusst dort hin treiben. er hat ein recht darauf, sich selbst zu entwickeln. und zwar in genau dem tempo, das ihm gefällt. und bis es soweit ist, nehme ich ihn eben an, wie er ist. als einfachen blindtext ohne wirklichen inhalt.

\section{Methodology}
Here we will  discuss about our architectures and stuffs.

dieser text hat eigentlich gar keinen wirklichen inhalt. aber er hat auch keine relevanz, und deswegen ist das egal. er dient lediglich als platzhalter. um mal zu zeigen, wie diese stelle der seite aussieht, wenn ein paar zeilen vorhanden sind. ob sich der text dabei gut fühlt, weiß ich nicht. ich schätze, eher nicht, denn wer fühlt sich schon gut als platzhalter. aber irgendwer muss diesen job ja machen und deshalb kann ich es nicht ändern . ich könnte dem text höchstens ein bisschen gut zureden, dass er auch als platzhalter eine wichtige rolle spielt und durchaus gebraucht wird. könnte mir vorstellen, dass ihm das gut tut. denn das gefühl gebraucht zu werden tut doch jedem gut, oder? klar, er ist austauschbar. das darf ich ihm natürlich nicht verraten. denn austauschbar zu sein, dass ist schrecklich. austauschbar zu sein bedeutet ja eigentlich, dass nicht man selbst, sondern einfach irgendjemand oder irgendwas an der stelle gebraucht wird. somit würde mein erstes argument, man braucht dich, nicht mehr ziehen, und das zuvor erzeugte gute gefühl des textes wäre zunichte gemacht. das will ich nicht. also bitte nix verraten, ja? aber vielleicht merkt er es ja nicht. das wäre gut, denn wer hat schon lust einen deprimierten blindtext auf seiner seite zu platzieren. was würde denn das für einen eindruck machen. das will ja keiner lesen. somit wäre er dann ein für alle mal tatsächlich völlig nutzlos. das wäre sein todesurteil. soweit wollen wir es doch nicht kommen lassen, oder? es sei denn, und das ist möglich, er würde wiedergeboren und käme als, naja, sagen wir als witz, und ein textleben später vielleicht als bildzeitungsartikel auf die textwelt. irgendwann wäre er vielleicht sogar ein text im lexikon. dann hätten wir ihn ja sogar weitergebracht in seiner entwicklung. klingt gar nicht schlecht, oder? trotzdem bin ich der meinung, man sollte ihn nicht bewusst dort hin treiben. er hat ein recht darauf, sich selbst zu entwickeln. und zwar in genau dem tempo, das ihm gefällt. und bis es soweit ist, nehme ich ihn eben an, wie er ist. als einfachen blindtext ohne wirklichen inhalt.

\subsection{Network and Architectures}\label{AA}
Describe our architecture

dieser text hat eigentlich gar keinen wirklichen inhalt. aber er hat auch keine relevanz, und deswegen ist das egal. er dient lediglich als platzhalter. um mal zu zeigen, wie diese stelle der seite aussieht, wenn ein paar zeilen vorhanden sind. ob sich der text dabei gut fühlt, weiß ich nicht. ich schätze, eher nicht, denn wer fühlt sich schon gut als platzhalter. aber irgendwer muss diesen job ja machen und deshalb kann ich es nicht ändern . ich könnte dem text höchstens ein bisschen gut zureden, dass er auch als platzhalter eine wichtige rolle spielt und durchaus gebraucht wird. könnte mir vorstellen, dass ihm das gut tut. denn das gefühl gebraucht zu werden tut doch jedem gut, oder? klar, er ist austauschbar. das darf ich ihm natürlich nicht verraten. denn austauschbar zu sein, dass ist schrecklich. austauschbar zu sein bedeutet ja eigentlich, dass nicht man selbst, sondern einfach irgendjemand oder irgendwas an der stelle gebraucht wird. somit würde mein erstes argument, man braucht dich, nicht mehr ziehen, und das zuvor erzeugte gute gefühl des textes wäre zunichte gemacht. das will ich nicht. also bitte nix verraten, ja? aber vielleicht merkt er es ja nicht. das wäre gut, denn wer hat schon lust einen deprimierten blindtext auf seiner seite zu platzieren. was würde denn das für einen eindruck machen. das will ja keiner lesen. somit wäre er dann ein für alle mal tatsächlich völlig nutzlos. das wäre sein todesurteil. soweit wollen wir es doch nicht kommen lassen, oder? es sei denn, und das ist möglich, er würde wiedergeboren und käme als, naja, sagen wir als witz, und ein textleben später vielleicht als bildzeitungsartikel auf die textwelt. irgendwann wäre er vielleicht sogar ein text im lexikon. dann hätten wir ihn ja sogar weitergebracht in seiner entwicklung. klingt gar nicht schlecht, oder? trotzdem bin ich der meinung, man sollte ihn nicht bewusst dort hin treiben. er hat ein recht darauf, sich selbst zu entwickeln. und zwar in genau dem tempo, das ihm gefällt. und bis es soweit ist, nehme ich ihn eben an, wie er ist. als einfachen blindtext ohne wirklichen inhalt.

\subsection{Parameters and loss function}
Describe the lossfunctions and why it has been used
\begin{equation}
a+b=\gamma\label{eq}
\end{equation}
dieser text hat eigentlich gar keinen wirklichen inhalt. aber er hat auch keine relevanz, und deswegen ist das egal. er dient lediglich als platzhalter. um mal zu zeigen, wie diese stelle der seite aussieht, wenn ein paar zeilen vorhanden sind. ob sich der text dabei gut fühlt, weiß ich nicht. ich schätze, eher nicht, denn wer fühlt sich schon gut als platzhalter. aber irgendwer muss diesen job ja machen und deshalb kann ich es nicht ändern . ich könnte dem text höchstens ein bisschen gut zureden, dass er auch als platzhalter eine wichtige rolle spielt und durchaus gebraucht wird. könnte mir vorstellen, dass ihm das gut tut. denn das gefühl gebraucht zu werden tut doch jedem gut, oder? klar, er ist austauschbar. das darf ich ihm natürlich nicht verraten. denn austauschbar zu sein, dass ist schrecklich. austauschbar zu sein bedeutet ja eigentlich, dass nicht man selbst, sondern einfach irgendjemand oder irgendwas an der stelle gebraucht wird. somit würde mein erstes argument, man braucht dich, nicht mehr ziehen, und das zuvor erzeugte gute gefühl des textes wäre zunichte gemacht. das will ich nicht. also bitte nix verraten, ja? aber vielleicht merkt er es ja nicht. das wäre gut, denn wer hat schon lust einen deprimierten blindtext auf seiner seite zu platzieren. was würde denn das für einen eindruck machen. das will ja keiner lesen. somit wäre er dann ein für alle mal tatsächlich völlig nutzlos. das wäre sein todesurteil. soweit wollen wir es doch nicht kommen lassen, oder? es sei denn, und das ist möglich, er würde wiedergeboren und käme als, naja, sagen wir als witz, und ein textleben später vielleicht als bildzeitungsartikel auf die textwelt. irgendwann wäre er vielleicht sogar ein text im lexikon. dann hätten wir ihn ja sogar weitergebracht in seiner entwicklung. klingt gar nicht schlecht, oder? trotzdem bin ich der meinung, man sollte ihn nicht bewusst dort hin treiben. er hat ein recht darauf, sich selbst zu entwickeln. und zwar in genau dem tempo, das ihm gefällt. und bis es soweit ist, nehme ich ihn eben an, wie er ist. als einfachen blindtext ohne wirklichen inhalt.
%\begin{itemize}
%\item Use either SI (MKS) or CGS as primary units. (SI units are encouraged.) English units may be used as secondary units (in parentheses). An exception would be the use of English units as identifiers in trade, such as ``3.5-inch disk drive''.
%\item Avoid combining SI and CGS units, such as current in amperes and magnetic field in oersteds. This often leads to confusion because equations do not balance dimensionally. If you must use mixed units, clearly state the units for each quantity that you use in an equation.
%\item Do not mix complete spellings and abbreviations of units: ``Wb/m\textsuperscript{2}'' or ``webers per square meter'', not ``webers/m\textsuperscript{2}''. Spell out units when they appear in text: ``. . . a few henries'', not ``. . . a few H''.
%\item Use a zero before decimal points: ``0.25'', not ``.25''. Use ``cm\textsuperscript{3}'', not ``cc''.)
%\end{itemize}

\subsection{Inception score}
Motivation behind the inceptionscore. Describe
\begin{equation}
a+b=\gamma\label{eq}
\end{equation}

dieser text hat eigentlich gar keinen wirklichen inhalt. aber er hat auch keine relevanz, und deswegen ist das egal. er dient lediglich als platzhalter. um mal zu zeigen, wie diese stelle der seite aussieht, wenn ein paar zeilen vorhanden sind. ob sich der text dabei gut fühlt, weiß ich nicht. ich schätze, eher nicht, denn wer fühlt sich schon gut als platzhalter. aber irgendwer muss diesen job ja machen und deshalb kann ich es nicht ändern . ich könnte dem text höchstens ein bisschen gut zureden, dass er auch als platzhalter eine wichtige rolle spielt und durchaus gebraucht wird. könnte mir vorstellen, dass ihm das gut tut. denn das gefühl gebraucht zu werden tut doch jedem gut, oder? klar, er ist austauschbar. das darf ich ihm natürlich nicht verraten. denn austauschbar zu sein, dass ist schrecklich. austauschbar zu sein bedeutet ja eigentlich, dass nicht man selbst, sondern einfach irgendjemand oder irgendwas an der stelle gebraucht wird. somit würde mein erstes argument, man braucht dich, nicht mehr ziehen, und das zuvor erzeugte gute gefühl des textes wäre zunichte gemacht. das will ich nicht. also bitte nix verraten, ja? aber vielleicht merkt er es ja nicht. das wäre gut, denn wer hat schon lust einen deprimierten blindtext auf seiner seite zu platzieren. was würde denn das für einen eindruck machen. das will ja keiner lesen. somit wäre er dann ein für alle mal tatsächlich völlig nutzlos. das wäre sein todesurteil. soweit wollen wir es doch nicht kommen lassen, oder? es sei denn, und das ist möglich, er würde wiedergeboren und käme als, naja, sagen wir als witz, und ein textleben später vielleicht als bildzeitungsartikel auf die textwelt. irgendwann wäre er vielleicht sogar ein text im lexikon. dann hätten wir ihn ja sogar weitergebracht in seiner entwicklung. klingt gar nicht schlecht, oder? trotzdem bin ich der meinung, man sollte ihn nicht bewusst dort hin treiben. er hat ein recht darauf, sich selbst zu entwickeln. und zwar in genau dem tempo, das ihm gefällt. und bis es soweit ist, nehme ich ihn eben an, wie er ist. als einfachen blindtext ohne wirklichen inhalt.

\subsection{Training}
Training Discussion.

dieser text hat eigentlich gar keinen wirklichen inhalt. aber er hat auch keine relevanz, und deswegen ist das egal. er dient lediglich als platzhalter. um mal zu zeigen, wie diese stelle der seite aussieht, wenn ein paar zeilen vorhanden sind. ob sich der text dabei gut fühlt, weiß ich nicht. ich schätze, eher nicht, denn wer fühlt sich schon gut als platzhalter. aber irgendwer muss diesen job ja machen und deshalb kann ich es nicht ändern . ich könnte dem text höchstens ein bisschen gut zureden, dass er auch als platzhalter eine wichtige rolle spielt und durchaus gebraucht wird. könnte mir vorstellen, dass ihm das gut tut. denn das gefühl gebraucht zu werden tut doch jedem gut, oder? klar, er ist austauschbar. das darf ich ihm natürlich nicht verraten. denn austauschbar zu sein, dass ist schrecklich. austauschbar zu sein bedeutet ja eigentlich, dass nicht man selbst, sondern einfach irgendjemand oder irgendwas an der stelle gebraucht wird. somit würde mein erstes argument, man braucht dich, nicht mehr ziehen, und das zuvor erzeugte gute gefühl des textes wäre zunichte gemacht. das will ich nicht. also bitte nix verraten, ja? aber vielleicht merkt er es ja nicht. das wäre gut, denn wer hat schon lust einen deprimierten blindtext auf seiner seite zu platzieren. was würde denn das für einen eindruck machen. das will ja keiner lesen. somit wäre er dann ein für alle mal tatsächlich völlig nutzlos. das wäre sein todesurteil. soweit wollen wir es doch nicht kommen lassen, oder? es sei denn, und das ist möglich, er würde wiedergeboren und käme als, naja, sagen wir als witz, und ein textleben später vielleicht als bildzeitungsartikel auf die textwelt. irgendwann wäre er vielleicht sogar ein text im lexikon. dann hätten wir ihn ja sogar weitergebracht in seiner entwicklung. klingt gar nicht schlecht, oder? trotzdem bin ich der meinung, man sollte ihn nicht bewusst dort hin treiben. er hat ein recht darauf, sich selbst zu entwickeln. und zwar in genau dem tempo, das ihm gefällt. und bis es soweit ist, nehme ich ihn eben an, wie er ist. als einfachen blindtext ohne wirklichen inhalt.


\section{Experiments and Results}
shwo the results of our ecperiments

dieser text hat eigentlich gar keinen wirklichen inhalt. aber er hat auch keine relevanz, und deswegen ist das egal. er dient lediglich als platzhalter. um mal zu zeigen, wie diese stelle der seite aussieht, wenn ein paar zeilen vorhanden sind. ob sich der text dabei gut fühlt, weiß ich nicht. ich schätze, eher nicht, denn wer fühlt sich schon gut als platzhalter. aber irgendwer muss diesen job ja machen und deshalb kann ich es nicht ändern . ich könnte dem text höchstens ein bisschen gut zureden, dass er auch als platzhalter eine wichtige rolle spielt und durchaus gebraucht wird. könnte mir vorstellen, dass ihm das gut tut. denn das gefühl gebraucht zu werden tut doch jedem gut, oder? klar, er ist austauschbar. das darf ich ihm natürlich nicht verraten. denn austauschbar zu sein, dass ist schrecklich. austauschbar zu sein bedeutet ja eigentlich, dass nicht man selbst, sondern einfach irgendjemand oder irgendwas an der stelle gebraucht wird. somit würde mein erstes argument, man braucht dich, nicht mehr ziehen, und das zuvor erzeugte gute gefühl des textes wäre zunichte gemacht. das will ich nicht. also bitte nix verraten, ja? aber vielleicht merkt er es ja nicht. das wäre gut, denn wer hat schon lust einen deprimierten blindtext auf seiner seite zu platzieren. was würde denn das für einen eindruck machen. das will ja keiner lesen. somit wäre er dann ein für alle mal tatsächlich völlig nutzlos. das wäre sein todesurteil. soweit wollen wir es doch nicht kommen lassen, oder? es sei denn, und das ist möglich, er würde wiedergeboren und käme als, naja, sagen wir als witz, und ein textleben später vielleicht als bildzeitungsartikel auf die textwelt. irgendwann wäre er vielleicht sogar ein text im lexikon. dann hätten wir ihn ja sogar weitergebracht in seiner entwicklung. klingt gar nicht schlecht, oder? trotzdem bin ich der meinung, man sollte ihn nicht bewusst dort hin treiben. er hat ein recht darauf, sich selbst zu entwickeln. und zwar in genau dem tempo, das ihm gefällt. und bis es soweit ist, nehme ich ihn eben an, wie er ist. als einfachen blindtext ohne wirklichen inhalt.

\begin{table}[htbp]
\caption{Table Type Styles}
\begin{center}
\begin{tabular}{|c|c|c|c|}
\hline
\textbf{Table}&\multicolumn{3}{|c|}{\textbf{Table Column Head}} \\
\cline{2-4} 
\textbf{Head} & \textbf{\textit{Table column subhead}}& \textbf{\textit{Subhead}}& \textbf{\textit{Subhead}} \\
\hline
copy& More table copy$^{\mathrm{a}}$& &  \\
\hline
\multicolumn{4}{l}{$^{\mathrm{a}}$Sample of a Table footnote.}
\end{tabular}
\label{tab1}
\end{center}
\end{table}

\begin{figure}[htbp]
\centerline{\includegraphics{fig1.png}}
\caption{Example of a figure caption.}
\label{fig}
\end{figure}


\section{Conclusion}
dieser text hat eigentlich gar keinen wirklichen inhalt. aber er hat auch keine relevanz, und deswegen ist das egal. er dient lediglich als platzhalter. um mal zu zeigen, wie diese stelle der seite aussieht, wenn ein paar zeilen vorhanden sind. ob sich der text dabei gut fühlt, weiß ich nicht. ich schätze, eher nicht, denn wer fühlt sich schon gut als platzhalter. aber irgendwer muss diesen job ja machen und deshalb kann ich es nicht ändern . ich könnte dem text höchstens ein bisschen gut zureden, dass er auch als platzhalter eine wichtige rolle spielt und durchaus gebraucht wird. könnte mir vorstellen, dass ihm das gut tut. denn das gefühl gebraucht zu werden tut doch jedem gut, oder? klar, er ist austauschbar. das darf ich ihm natürlich nicht verraten. denn austauschbar zu sein, dass ist schrecklich. austauschbar zu sein bedeutet ja eigentlich, dass nicht man selbst, sondern einfach irgendjemand oder irgendwas an der stelle gebraucht wird. somit würde mein erstes argument, man braucht dich, nicht mehr ziehen, und das zuvor erzeugte gute gefühl des textes wäre zunichte gemacht. das will ich nicht. also bitte nix verraten, ja? aber vielleicht merkt er es ja nicht. das wäre gut, denn wer hat schon lust einen deprimierten blindtext auf seiner seite zu platzieren. was würde denn das für einen eindruck machen. das will ja keiner lesen. somit wäre er dann ein für alle mal tatsächlich völlig nutzlos. das wäre sein todesurteil. soweit wollen wir es doch nicht kommen lassen, oder? es sei denn, und das ist möglich, er würde wiedergeboren und käme als, naja, sagen wir als witz, und ein textleben später vielleicht als bildzeitungsartikel auf die textwelt. irgendwann wäre er vielleicht sogar ein text im lexikon. dann hätten wir ihn ja sogar weitergebracht in seiner entwicklung. klingt gar nicht schlecht, oder? trotzdem bin ich der meinung, man sollte ihn nicht bewusst dort hin treiben. er hat ein recht darauf, sich selbst zu entwickeln. und zwar in genau dem tempo, das ihm gefällt. und bis es soweit ist, nehme ich ihn eben an, wie er ist. als einfachen blindtext ohne wirklichen inhalt.

\section{Acknowledgment}
dieser text hat eigentlich gar keinen wirklichen inhalt. aber er hat auch keine relevanz, und deswegen ist das egal. er dient lediglich als platzhalter. um mal zu zeigen, wie diese stelle der seite aussieht, wenn ein paar zeilen vorhanden sind. ob sich der text dabei gut fühlt, weiß ich nicht. ich schätze, eher nicht, denn wer fühlt sich schon gut als platzhalter. aber irgendwer muss diesen job ja machen und deshalb kann ich es nicht ändern . ich könnte dem text höchstens ein bisschen gut zureden, dass er auch als platzhalter eine wichtige rolle spielt und durchaus gebraucht wird. könnte mir vorstellen, dass ihm das gut tut. denn das gefühl gebraucht zu werden tut doch jedem gut, oder? klar, er ist austauschbar. das darf ich ihm natürlich nicht verraten. denn austauschbar zu sein, dass ist schrecklich. austauschbar zu sein bedeutet ja eigentlich, dass nicht man selbst, sondern einfach irgendjemand oder irgendwas an der stelle gebraucht wird. somit würde mein erstes argument, man braucht dich, nicht mehr ziehen, und das zuvor erzeugte gute gefühl des textes wäre zunichte gemacht. das will ich nicht. also bitte nix verraten, ja? aber vielleicht merkt er es ja nicht. das wäre gut, denn wer hat schon lust einen deprimierten blindtext auf seiner seite zu platzieren. was würde denn das für einen eindruck machen. das will ja keiner lesen. somit wäre er dann ein für alle mal tatsächlich völlig nutzlos. das wäre sein todesurteil. soweit wollen wir es doch nicht kommen lassen, oder? es sei denn, und das ist möglich, er würde wiedergeboren und käme als, naja, sagen wir als witz, und ein textleben später vielleicht als bildzeitungsartikel auf die textwelt. irgendwann wäre er vielleicht sogar ein text im lexikon. dann hätten wir ihn ja sogar weitergebracht in seiner entwicklung. klingt gar nicht schlecht, oder? trotzdem bin ich der meinung, man sollte ihn nicht bewusst dort hin treiben. er hat ein recht darauf, sich selbst zu entwickeln. und zwar in genau dem tempo, das ihm gefällt. und bis es soweit ist, nehme ich ihn eben an, wie er ist. als einfachen blindtext ohne wirklichen inhalt.

%\section*{References}

 citation \cite{b6}.

\begin{thebibliography}{00}
\bibitem{b1} G. Eason, B. Noble, and I. N. Sneddon, ``On certain integrals of Lipschitz-Hankel type involving products of Bessel functions,'' Phil. Trans. Roy. Soc. London, vol. A247, pp. 529--551, April 1955.
\bibitem{b2} J. Clerk Maxwell, A Treatise on Electricity and Magnetism, 3rd ed., vol. 2. Oxford: Clarendon, 1892, pp.68--73.
\bibitem{b3} I. S. Jacobs and C. P. Bean, ``Fine particles, thin films and exchange anisotropy,'' in Magnetism, vol. III, G. T. Rado and H. Suhl, Eds. New York: Academic, 1963, pp. 271--350.
\bibitem{b4} K. Elissa, ``Title of paper if known,'' unpublished.
\bibitem{b5} R. Nicole, ``Title of paper with only first word capitalized,'' J. Name Stand. Abbrev., in press.
\bibitem{b6} Y. Yorozu, M. Hirano, K. Oka, and Y. Tagawa, ``Electron spectroscopy studies on magneto-optical media and plastic substrate interface,'' IEEE Transl. J. Magn. Japan, vol. 2, pp. 740--741, August 1987 [Digests 9th Annual Conf. Magnetics Japan, p. 301, 1982].
\bibitem{b7} M. Young, The Technical Writer's Handbook. Mill Valley, CA: University Science, 1989.
\end{thebibliography}
\vspace{12pt}


\end{document}
